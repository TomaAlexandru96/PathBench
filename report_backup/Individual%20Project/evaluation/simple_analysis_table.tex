\pagebreak

\begin{table}[] 
\footnotesize 
\small
\centering

\begin{tabular}{|cc|c|c|c|c|c|}
\hline
\multicolumn{2}{|c|}{\textbf{Nr.}} & \textbf{Success Rate} & \textbf{Distance} & \textbf{Time} & \textbf{Distance Left}\\
\hline
\hline
\multicolumn{2}{|c|}{\cellcolor{lightgray!20} \hyperref[tab: evalalgorithms]{0}} & 93.33\% (I: 0\%) & 39.55 (A*: 39.55) (I: 0\%) & 0.055s & 2.36\\
\hline
\hline
\multicolumn{2}{|c|}{\cellcolor{red!40} \hyperref[tab: evalalgorithms]{1}} & 46.67\% (I: -49.99\%) & 35.09 (A*: 34.06) (I: -3.02\%) & 0.1342s & 9.93\\
\hline
\multicolumn{2}{|c|}{\cellcolor{red!20} \hyperref[tab: evalalgorithms]{2}} & 46.67\% (I: -49.99\%) & 44.04 (A*: 36.77) (I: -19.77\%) & 0.1728s & 12.52\\
\hline
\multicolumn{2}{|c|}{\cellcolor{red!20} \hyperref[tab: evalalgorithms]{3}} & 60.0\% (I: -35.71\%) & 45.44 (A*: 38.97) (I: -16.6\%) & 0.1328s & 6.29\\
\hline
\multicolumn{2}{|c|}{\cellcolor{red!20} \hyperref[tab: evalalgorithms]{4}} & 50.0\% (I: -46.43\%) & 37.43 (A*: 35.07) (I: -6.73\%) & 0.1051s & 9.45\\
\hline
\multicolumn{2}{|c|}{\cellcolor{red!20} \hyperref[tab: evalalgorithms]{5}} & 70.0\% (I: -25.0\%) & 43.88 (A*: 38.59) (I: -13.71\%) & 0.1166s & 6.47\\
\hline
\hline
\multicolumn{2}{|c|}{\cellcolor{blue!20} \hyperref[tab: evalalgorithms]{6}} & 40.0\% (I: -57.14\%) & 28.15 (A*: 27.14) (I: -3.72\%) & 0.1047s & 14.24\\
\hline
\multicolumn{2}{|c|}{\cellcolor{blue!40} \hyperref[tab: evalalgorithms]{7}} & 43.33\% (I: -53.57\%) & 37.96 (A*: 33.79) (I: -12.34\%) & 0.1399s & 13.59\\
\hline
\multicolumn{2}{|c|}{\cellcolor{blue!20} \hyperref[tab: evalalgorithms]{8}} & 56.67\% (I: -39.28\%) & 44.16 (A*: 39.03) (I: -13.14\%) & 0.1579s & 10.52\\
\hline
\multicolumn{2}{|c|}{\cellcolor{blue!20} \hyperref[tab: evalalgorithms]{9}} & 56.67\% (I: -39.28\%) & 42.19 (A*: 36.86) (I: -14.46\%) & 0.1513s & 7.68\\
\hline
\multicolumn{2}{|c|}{\cellcolor{blue!20} \hyperref[tab: evalalgorithms]{10}} & 60.0\% (I: -35.71\%) & 40.58 (A*: 36.29) (I: -11.82\%) & 0.1452s & 7.38\\
\hline
\hline
\multicolumn{2}{|c|}{\cellcolor{orange!40} \hyperref[tab: evalalgorithms]{11}} & 83.33\% (I: -10.71\%) & 43.22 (A*: 40.33) (I: -7.17\%) & 1.3548s & 1.63\\
\hline
\hline
\multicolumn{1}{|M{0.15cm}}{\cellcolor{cyan!40}} & \multicolumn{1}{M{0.15cm}|}{\cellcolor{blue!40} \hspace*{-0.5cm}\hyperref[tab: evalalgorithms]{12}} & 93.33\% (I: 0\%) & 45.76 (A*: 39.55) (I: -15.7\%) & 0.407s & 0.95\\
\hline
\multicolumn{1}{|M{0.15cm}}{\cellcolor{cyan!40}} & \multicolumn{1}{M{0.15cm}|}{\cellcolor{red!40} \hspace*{-0.5cm}\hyperref[tab: evalalgorithms]{13}} & 93.33\% (I: 0\%) & 48.95 (A*: 39.55) (I: -23.77\%) & 0.3171s & 0.62\\
\hline
\multicolumn{1}{|M{0.15cm}}{\cellcolor{cyan!40}} & \multicolumn{1}{M{0.15cm}|}{\cellcolor{orange!40} \hspace*{-0.5cm}\hyperref[tab: evalalgorithms]{14}} & 93.33\% (I: 0\%) & 44.94 (A*: 39.55) (I: -13.63\%) & 2.8274s & 0.64\\
\hline
\end{tabular}


\bigskip

\begin{tabular}{|cc|c|c|}
\hline
\multicolumn{2}{|c|}{\textbf{Nr.}} & \textbf{Pick Ratio}\\
\hline
\hline
\multicolumn{2}{|c|}{\cellcolor{orange!40} \hyperref[tab: evalalgorithms]{11}} & [23.33, 26.67, 10.0, 3.33, 3.33, 6.67, 10.0, 10.0, 0.0, 6.67]\%\\
\hline
\hline
\multicolumn{1}{|M{0.15cm}}{\cellcolor{cyan!40}} & \multicolumn{1}{M{0.15cm}|}{\cellcolor{orange!40} \hspace*{-0.5cm}\hyperref[tab: evalalgorithms]{14}} & [51.31, 21.02, 12.66, 3.43, 0.67, 0.0, 7.9, 1.67, 0.24, 1.11]\%\\
\hline
\end{tabular}


\bigskip

\begin{tabular}{|cc|c|c|c|c|M{3cm}|}
\hline
\multicolumn{2}{|c|}{\textbf{Nr.}} & \textbf{GK Improvement} & \textbf{GK Distance} & \textbf{GK Distance Left} & \textbf{WP} & \textbf{WP In-Between Distance}\\
\hline
\hline
\multicolumn{1}{|M{0.15cm}}{\cellcolor{cyan!40}} & \multicolumn{1}{M{0.15cm}|}{\cellcolor{blue!40} \hspace*{-0.5cm}\hyperref[tab: evalalgorithms]{12}} & 73.74\% & 30.97 & 11.08 & 5.03 & 10.12\\
\hline
\multicolumn{1}{|M{0.15cm}}{\cellcolor{cyan!40}} & \multicolumn{1}{M{0.15cm}|}{\cellcolor{red!40} \hspace*{-0.5cm}\hyperref[tab: evalalgorithms]{13}} & 86.51\% & 44.63 & 5.51 & 4.3 & 14.14\\
\hline
\multicolumn{1}{|M{0.15cm}}{\cellcolor{cyan!40}} & \multicolumn{1}{M{0.15cm}|}{\cellcolor{orange!40} \hspace*{-0.5cm}\hyperref[tab: evalalgorithms]{14}} & 97.0\% & 49.23 & 1.3 & 4.1 & 15.91\\
\hline
\end{tabular}


\bigskip

\begin{tabular}{|cc|c|c|c|c|c|}
\hline
\multicolumn{2}{|c|}{\textbf{Nr.}} & \textbf{Total Search} & \textbf{Total Fringe} & \textbf{Session Search} & \textbf{Session Fringe}\\
\hline
\hline
\multicolumn{2}{|c|}{\cellcolor{lightgray!20} \hyperref[tab: evalalgorithms]{0}} & 9.47\% & 2.86\% & 9.47\% & 2.86\%\\
\hline
\hline
\multicolumn{1}{|M{0.15cm}}{\cellcolor{cyan!40}} & \multicolumn{1}{M{0.15cm}|}{\cellcolor{blue!40} \hspace*{-0.5cm}\hyperref[tab: evalalgorithms]{12}} & 6.51\% (I: 31.26\%) & 2.75\% (I: 3.85\%) & 1.49\% (I: 84.27\%) & 0.71\% (I: 75.17\%)\\
\hline
\multicolumn{1}{|M{0.15cm}}{\cellcolor{cyan!40}} & \multicolumn{1}{M{0.15cm}|}{\cellcolor{red!40} \hspace*{-0.5cm}\hyperref[tab: evalalgorithms]{13}} & 6.61\% (I: 30.2\%) & 2.69\% (I: 5.94\%) & 1.77\% (I: 81.31\%) & 0.84\% (I: 70.63\%)\\
\hline
\multicolumn{1}{|M{0.15cm}}{\cellcolor{cyan!40}} & \multicolumn{1}{M{0.15cm}|}{\cellcolor{orange!40} \hspace*{-0.5cm}\hyperref[tab: evalalgorithms]{14}} & 5.82\% (I: 38.54\%) & 2.68\% (I: 6.29\%) & 1.78\% (I: 81.2\%) & 0.89\% (I: 68.88\%)\\
\hline
\end{tabular}


\caption{\textbf{Analyser} simple analysis on 30 maps (10 uniform random fill maps, 10 block maps, 10 house maps). All experiments will have the same structure as this figure. The statistics are described in Tables \ref{tab: a_testing_tab}, \ref{tab: LSTM Bagging Planner testing_tab} and \ref{tab: WayPointNavigation testing_tab} and are all averaged. The parenthesis value containing the A*: prefix is the A* results that were run only on the filtered succeeded paths associated with the row run. The parenthesis value containing the I: prefix is the improvement ratio against A* (positive is better improvement and negative is degradation).}
\label{tab: eval_simple_analysis} 
\end{table}

The simple analysis results (See Table \ref{tab: eval_simple_analysis}) show that 2 maps (from 30) do not have a solution (given by A* success rate). We can notice that the best Online LSTM Planner was Algorithm \hyperref[tab: evalalgorithms]{5} with a 70\% (I: -25\%) success rate and $-13.71\%$ distance improvement. The Online LSTM Planner which has been trained on the same type of map as \cite{nicola2018lstm} (Algorithm \hyperref[tab: evalalgorithms]{1}) has a poorer success rate of 46.67\% (I: -49.99\%), but higher distance improvement rate $-3.02\%$ (distance improvement is only computed on successful paths). By comparing these results with the results from \cite{nicola2018lstm} (See Table \ref{tab: nicola_results}; Success Rate: 82\% (I: -18\%), Distance: (I: -16.23\%)), we can observe that we have a general lower success rate, but higher distance improvement rate. The best CAE Online LSTM Planner is Algorithm \hyperref[tab: evalalgorithms]{10} with 60.0\% (-35.71\%) success rate and $-11.82\%$ distance improvement. Algorithm \hyperref[tab: evalalgorithms]{7} (which was trained on the same type of generated maps as \cite{inoue2019robot}: block maps) has poorer success rate 43.33\% (I: -53.57\%) and lower distance improvement $-12.34\%$. Overall, the results show that the Online LSTM and CAE Online LSTM have poorer success rate and insignificantly higher distance improvement rate than \cite{nicola2018lstm}). It should be noted that we use a higher number of environment maps (30) than \cite{nicola2018lstm} (10) and \cite{inoue2019robot} (10) in order to ensure the generalisation property of our solutions. By using the LSTM Bagging Planner (Algorithm \hyperref[tab: evalalgorithms]{11}) we drastically increase the performance of the algorithms to 83.33\% (I:-10.71\%) (higher than \cite{nicola2018lstm}; 82\% (I: -18\%)) and decrease the distance improvement rate even further (-7.17\% < -16.23\%). The Global Way-point LSTM Planners (Algorithms \hyperref[tab: evalalgorithms]{12}, \hyperref[tab: evalalgorithms]{13} and \hyperref[tab: evalalgorithms]{14}) have the same success rate as A* (due to the fact that we use A* as our local kernel) and lower overall distance rate improvement. The proposed solution (Algorithm \hyperref[tab: evalalgorithms]{14}) has a reasonable distance rate improvement of $-13.63\%$ compared to the other algorithms.

The pick ratio of the LSTM Bagging Planner (Algorithm \hyperref[tab: evalalgorithms]{11}) is well distributed compared to the proposed solution (Algorithm \hyperref[tab: evalalgorithms]{14}). The proposed solution (Algorithm \hyperref[tab: evalalgorithms]{14}) uses the kernel priority system described in Chapter \ref{sec: methods} (\hyperref[sec: methods]{Methods}) (which can be noticed from the statistics; kernels lose pick percentage the further away they are from the head of the list). This shows that the priority of the kernels is essential to ensure the generation of a good path.

The third table shows that the proposed solution (Algorithm \hyperref[tab: evalalgorithms]{14}) has a really high GK Improvement rate compared to the other Global Way-point LSTM Planners (Algorithms \hyperref[tab: evalalgorithms]{12} and \hyperref[tab: evalalgorithms]{13}), a high GK Distance (which is directly correlated to the GK Improvement), a great GK Distance Left, a lower WP count and a higher WP In-Between Distance. The GK Improvement rate and GK Distance show that the algorithm has suggested good way-points which accounted for the majority of the path journey. However, a higher GK Distance than the total travelled distance is a sign of oscillation which unnecessarily boosts the GK Improvement metric. However, because the difference between the GK Distance and the total Distance is small, the boost is reduced and almost insignificant. A low number of way-points with a higher way-point in-between distance is preferred over a larger number of way-points with a smaller way-point in-between distance due to the fact that we use the local kernel to optimise the path between the way-points. The GK Distance Left is really small which shows that the global kernel got lost near the goal. This metric is less useful when the environment resembles a maze as we might have to go around a long wall to get to the goal, but in our case, the environments are eligible for using the GK Distance Left metric.

The final table compares the used memory of the Global Way-point LSTM Planners against A*. We provide the total search and fringe space to get a general idea of the way-point efficiency, but only the session search and fringe space are taken into account as the memory is bounded by a single session. We can see that the search space is drastically reduced for all Global Way-point LSTM Planners compared to A*.

All algorithms have worse time performance than A*, but the LSTM Bagging Planner (Algorithm \hyperref[tab: evalalgorithms]{11}) and the proposed solution (Algorithm \hyperref[tab: evalalgorithms]{14}) have the highest times due to the implementation issues described in previous sections.

For the following complex analysis phase we are only going to mention the most noticeable changes (compared to the simple analysis) for the following algorithms: the Online LSTM Planner trained on the same type of maps as \cite{nicola2018lstm} (Algorithm \hyperref[tab: evalalgorithms]{1}), the CAE Online LSTM Planner trained on the same type of maps as \cite{inoue2019robot} (Algorithm \hyperref[tab: evalalgorithms]{7}), the LSTM Bagging Planner (Algorithm \hyperref[tab: evalalgorithms]{11}) and the proposed solution (Algorithm \hyperref[tab: evalalgorithms]{14}). It should be noted that the generated maps have not been used in training and can be considered unknown as well.

\newpage