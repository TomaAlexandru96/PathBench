\chapter{Conclusion} \label{Conclusion}

% As a frame of reference \cite{nicola2018lstm} states that they have achieved 68\% success rate on Dragon Age maps, 26\% success rate on mazes with corridor 8 steps long and 82\% success rate on random filled maps, \cite{inoue2019robot} states that they have achieved over 98\% in 8 our 10 environment maps.

In this chapter, we are going to summarise our findings and address future work.

\section{Summary}

In conclusion, we have successfully applied Machine Learning methods to the pathfinding problem. We have developed a hybrid solution which combines classic and ML methods to solve the path planning problem while maintaining support in full and partial knowledge environments, reducing the memory load compared to A*, generalising well in unknown environments and theoretically satisfying the real-world industrial application requirements.

%We have created an extensible simulation platform for visualising robotic path planners, a map generator, a training environment for machine learning path planning models and an analyser tool to study the performance of the current and proposed solutions. We have developed a flexible algorithm which boosts the performance of the LSTM solutions by combining multiple weak learners into a way-point suggestion algorithm and bounding the number of session iterations. The algorithm trades off path optimality to reduce the memory usage and gain flexibility: support for randomized kinodynamic planning, support for non-holonomic constraints, 3D scaling and support for highly dynamic environments. Lastly, the current solution solves the long corridor issue from paper \cite{nicola2018lstm} to some extent, but still has issues when going around unusually shaped obstacles.

The summary of the contributions is presented as follows (the source code can be found at \url{https://gitlab.doc.ic.ac.uk/ait15/individual-project}): 

\begin{itemize}
    \item Algorithmic solutions: \begin{itemize}
        \item \textbf{CAE Online LSTM Planner} - We have managed to fix the long corridor issue from \cite{nicola2018lstm} to some extent by using a CAE network architecture
        \item \textbf{LSTM Bagging Planner} - We have improved the overall performance of the Online LSTM and CAE Online LSTM Planners by combining their solutions
        \item \textbf{Global Way-point LSTM Planner} - We have shown that the proposed solution theoretically achieves lower average case time and space complexity than A* (in 2D and 3D environments), is online, supports partial knowledge environments, reduces the memory load compared to A*, is robust to unknown environments and theoretically satisfies the real-world industrial application requirements (depending on the choice of local kernel)
    \end{itemize}
    \item PathBench: 
    \begin{itemize}
        \item \textbf{Simulator} - We have built a simulator to visualise path planning algorithms. Moreover, the simulator abstracts the interactions between the robot and the environment for faster development and testing of new solutions
        \item \textbf{Generator} - We have developed methods for generating synthetic ML training datasets
        \item \textbf{Trainer} - We have built a training environment to boost the productivity of testing new ML architectures
        \item \textbf{Analyser} - We have developed an analyser tool to create custom benchmarking statistics in order to assess the performance of the proposed solution
        \item \textbf{ROS Real-time Extension} - We have added support for real-world simulation by implementing an updatable map environment which is compatible with the \textit{gmapping} \textit{ROS} package
    \end{itemize}
    \item Theoretic and real-world evaluations: \begin{itemize}
        \item \textbf{Complexity and Theoretical Analysis} - We have stated the theoretical worst (and average) case time and space complexities for all discussed algorithms. We have shown that the proposed solution theoretically achieves lower average case time and space complexity than A*
        \item \textbf{Empirical Methods} - We have practically evaluated the proposed solutions by running empirical routines using custom benchmarking statistics. We have proved that the Global Way-point LSTM Planner significantly reduces the memory load compared to A* and that it generalises well in unknown environments
        \item \textbf{Real-world Evaluation} - We have tested the performance of the proposed solution on real-world occupancy grid maps generated by real-world robots. We have implemented the proposed solution on a real-world robot and tested it at Imperial College London. We have proved that the Global Way-point LSTM Planner supports partial knowledge environments
    \end{itemize}
\end{itemize}

% maybe write about RRT instead of A*

\section{Future Work}

% The current solution solves the long corridor issue from paper \cite{nicola2018lstm} to some extent, but different machine learning architectures should be considered to fully overcome this issue and more synthetic and real training data should be acquired to enhance the current machine learning solutions. We have proved that the proposed solution theoretically achieves better time and space complexity than A*, but it still needs refining in order to become a practical solution. 

We believe that more work can be done in the area of pathfinding ML methods to improve the overall performance of the Global Way-point LSTM Planner and potentially find more solutions:

\begin{itemize}
    \item \textbf{Parallelising Issue Fix} - This should be a top priority as it is a trivial performance boost that cannot be used due to the possible issues in the \textit{pytorch} framework
    \item \textbf{Path Refining Techniques} - The produced path is sometimes unnecessarily long and can be shortened by applying some path refining techniques (e.g. \cite{inoue2019robot} uses this kind of techniques to generate a high-quality path) such as a moving average. Not only that we reduce the distance by a significant amount, but we can also reduce the collision chance by pushing the path further away from the obstacles (if we use the moving average technique)
    \item \textbf{Advanced Synthetic Data Generation Techniques} - More generation methods should be investigated such as: Maze Generation, Cellular Automata Cave Generation and Generative Adversarial Networks (GANs) \cite{ml, Goodfellow-et-al-2016, russell2016artificial, lecun2015deep}
    \item \textbf{Real Datasets} - The problem with synthetic generated datasets is that the ML models might learn the generation procedure and will be biased when dealing with unknown environments (i.e. the generalisation property is deteriorated). Therefore, having a real dataset should improve the overall performance of the ML models and boost the generalisation property
    \item \textbf{Advanced Hyper-parameter Search} - As seen in the real-world experiments, the hyper-parameter choice (kernels, kernel priority and max global kernel iterations) is crucial. Moreover, we share the same issue with the training performance of ML methods. Therefore, more work should be done in this area by adopting different search strategies that improve the overall performance of the ML models such as: Grid Search, Random Search and optimisation techniques (Bayesian Optimisation, Gradient-based Optimisation, Line Search, Golden Section, Newton Methods, Lipschitz Optimisation) \cite{chong2013introduction}
    \item \textbf{New Machine Learning Models} - More ML methods should be investigated such as an LSTM network that uses Attention, Gated Recurrent Units (GRUs), PCA, GANs for generating datasets, Reinforcement Learning approaches (Deep Q-Networks (DQNs), Value Iteration Networks (VINs)) and many more \cite{ml, Goodfellow-et-al-2016, russell2016artificial, lecun2015deep}
    \item \textbf{Practical Evaluations for Theoretical Properties} - We have shown that the proposed solution has support for partial knowledge environments, reduces the memory load compared to A* and generalises well in unknown environments, but we have only theoretically proven the real-world industrial application requirements. Therefore, more work should be taken to empirically assess the theoretically proven properties
\end{itemize}